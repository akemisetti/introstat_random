\subsection{Sample variability}

\begin{example}{Because the percent of possums that are male where \var{Pop} is \var{other} is 67\%, does this mean the New South Wales and Queensland possum population is 67\% male?}
First consider a small sample. If 10 random possums were selected and 7 in 10 were male, does this mean that exactly 70\% of the possum population is male? Certainly not.

What if the sample size was 25 and 12 were male. Would it mean that exactly 48\% of possums in the population are male? Again, no.

So what would make this sample (of 58) large enough to make such a conclusion? Nothing. While the 67\% is probably somewhere near the actual percent of males, it is probably not exactly correct.
\end{example}

In reality, an estimate from a sample does not perfectly reflect the population. The proportion of males in the sample -- 67\% -- is just an \term{estimate} of the population. This begs the question: how closely do estimates reflect the population? Regrettably, this is not an easy question to answer, however, it will be formally answered (finally) in Chapter~\ref{classicalInference}. \\

\begin{tipBox}{\tipBoxTitle{Sample variability}
While a sample is useful to make conclusions about a population, a sample does not typically perfectly represent the population. Sample estimates usually have some uncertainty or error.}
\end{tipBox}

\begin{exercise}
There is one special case when the estimate is exact. Suppose there are 10,000 possums in all of New South Wales and you would like to know what proportion are male. How many of them must you sample to obtain the exact proportion of males in the population? (Hint: can you skip sampling any to know the exact proportion?)
\end{exercise}

