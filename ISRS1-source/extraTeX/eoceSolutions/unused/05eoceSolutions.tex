\documentclass[11pt]{article}
\usepackage{geometry, graphicx, amssymb, amsmath, epstopdf, multicol, multirow}
\geometry{letterpaper}
\DeclareGraphicsRule{.tif}{png}{.png}{`convert #1 `dirname #1`/`basename #1 .tif`.png}
\newcounter{eoceSolCh}
\setcounter{eoceSolCh}{0}
\newcommand{\eoceSolCh}[1]{
\refstepcounter{eoceSolCh}\noindent\textbf{\arabic{eoceSolCh}\hspace{2mm}#1}

\addvspace{2mm}

}
\newcounter{eoceSol}[eoceSolCh]
\newcommand{\eoceSol}[1]{\refstepcounter{eoceSol}\noindent\small\textbf{\arabic{eoceSolCh}.\arabic{eoceSol}}\hspace{2mm}#1\addtocounter{eoceSol}{1}

\addvspace{1mm}

}
\begin{document}


%%%%%%%%%%%%%%%%%%%%%%%

\setcounter{eoceSolCh}{4}
\eoceSolCh{Large sample inference}

%\begin{multicols}{2}

\eoceSol{(a) Hypothesis test for paired data (b) Independence is satisfied since we have a random sample that is less than 10\% of the population. Normality is satisfied since the sample size is large enough and the two samples are dependent. (c) $H_0: \mu_{diff} = 0$ (There is no difference in average daily high temperature between January 1, 1968 and January 1, 2008), $H_0: \mu_{diff} > 0$ (Average daily high temperature in January 1, 1968 was lower than average daily high temperature in January, 2008.) (d) Z = 1.60, p-value = 0.0548. (e) Fail to reject $H_0$. The data do not provide convincing evidence of temperature warming in the continental US. However it should be noted that the p-value is very close to 0.05. (f) Type II. If we made such an error and concluded that there isn't convincing evidence for temperature warming in the continental US, but in reality average temperature on January 1, 2008 is significantly higher than average temperature on January 1, 1968. (g) Yes.}

\eoceSol{(a) $H_0: \mu_{B} = \mu_{A} \rightarrow \mu_{B} - \mu_{A} = 0$ (The population mean of number of cigarettes smoked per day did not change after the Surgeon General's report), $H_A: \mu_1 > \mu_2 \rightarrow \mu_1 - \mu_2 > 0$ (The population mean of number of cigarettes smoked per day has decreased after the Surgeon General's report) (b) Independence is satisfied since we have a random sample that is less than 10\% of the population. Normality is satisfied since the sample size is large enough and the samples are independent. (c) Z = 1.89, p-value = 0.0294. (d) Reject $H_0$. There is sufficient evidence to suggest that the number of cigarettes smoked per day has decreased after the Surgeon General's report. (e) No, the result of the hypothesis test does not imply causation. (f) Type I.}

\eoceSol{(a) $H_0: \mu_{1990} = \mu_{2004} \rightarrow \mu_{1990} - \mu_{2004} = 0$ (Average math score in 1990 is equal to average math score in 2004), $H_A: \mu_1 > \mu_2 \rightarrow \mu_1 - \mu_2 > 0$ (Average math score in 1990 is different than average math score in 2004) (b) Independence is satisfied since we have a random sample that is less than 10\% of the population. Normality is satisfied since the sample size is large enough and the samples are independent. (c) Z = -1.46, p-value = 0.1442. (d) Fail to reject $H_0$. The data do not provide convincing evidence to suggest that the average score has changed between 1990 and 2004. (e) Type II.}

\eoceSol{(a) $H_0: \mu_M = \mu_W$, $H_A: \mu_M < \mu_W$. Z = -97.35, p-value $\approx$ 0. Reject $H_0$. There is sufficient evidence to suggest that average body fat percentage for women is higher. (b) Women on average tend to weigh less than men. Therefore their lean mass is bound to be lower than that of men.}

\eoceSol{(a) True. (b) False, doubling the sample size would decrease the standard error of the sample proportion only by a factor of $\sqrt{2}$. (c) True. (d) True. (e) False,  success-failure condition is not satisfied therefore the distribution of $\hat{p}$ is not nearly normal.}

\eoceSol{(a) False, a confidence interval is constructed to estimate the population proportion.. (b) True. (c) False, the confidence interval does not tell us what we might expect to see in another random sample. (d) True. (e) True.}

\eoceSol{(a) Sample statistic: proportion of people in the sample who have travelled abroad, $\hat{p} = 0.42$, population parameter: proportion of all students at this university who have travelled abroad, $p$. (b) Independence is satisfied since we have a random sample that is less than 10\% of the population. Normality is satisfied since the success-failure condition is met. (c) (0.34, 0.50) (d) We are 90\% confident that 34\% to 50\% of the students at this university have travelled abroad. (e) 90\% of random samples of 100 would produce a confidence interval that includes the true proportion of students at this university who have travelled abroad.}

\eoceSol{(a) All assumptions/conditions satisfied. (b) We are 95\% confident that 14.5\% to 25.5\% of all students at this university smoke. (c) 385 (d) We are 99\% confident that 73\% to 87\% of all students at this university do not smoke.}

\eoceSol{(a) $H_0: p = 0.35$, $H_A: p > 0.35$. Z = 1.47, p-value = 0.0708. Fail to reject $H_0$. The data do not provide convincing evidence to suggest that the proportion of students at this university who have travelled abroad has increased after the implementation of the study abroad program. (b) If in fact 35\% of students at this university have travelled abroad, the probability of getting a random sample of 100 students where more than 42\% have travelled abroad is 0.0708. (c) Yes.}

\eoceSol{(a) $H_0: p = 0.5$ (50\% of all marketing majors have a double major), $H_A: p > 0.5$ (More than 50\% of all marketing majors have a double major) (b) All assumptions/conditions satisfied. (c) Z = 2.24 (d) 0.025 (e) Reject $H_0$, the data do not provide convincing evidence to suggest that majority of marketing students have a double major.}

\eoceSol{(a) $H_0: p = 0.75$ (75\% of students at this college live at home), $H_A: p > 0.75$ (More than 65\% of students at this college live at home) (b) All assumptions/conditions satisfied. (c) Z = 2.89 (d) If in fact 75\% of student at this college lived at home, the probability of getting a random sample of 400 students where more than 325 live at home would be 0.0019. (e) Reject $H_0$, the data provide convincing evidence to suggest that more than 75\% of students at this college live at home.}

\eoceSol{(a) $H_0: p = 0.50$ (50\% of all independents are oppose the public option plan), $H_A:  p > 0.50$ (More than 50\% of all independents are oppose the public option plan) (b) All assumptions/conditions satisfied.(c) Z = 1.12 (d) If in fact 50\% of all independents oppose the public option plan, the probability of getting a sample of 783 independent where more than 52\% oppose the plan is 0.1314. (e) Fail to reject $H_0$, the data do not provide convincing evidence to suggest that more than half of all independents oppose the public option plan. (f) Yes.}

\eoceSol{$H_0: p = 0.5, H_A: p > 0.5$. All assumptions/conditions satisfied. Z = 2.91, p-value = 0.0018. Reject $H_0$, the data provide convincing evidence to suggest that the rate of correctly identifying a soda for these people is significantly better than just by random guessing.}

\eoceSol{$H_0: p = 0.3, H_A: p > 0.3$. All assumptions/conditions satisfied. Z = 1.89, p-value = 0.0294. Reject $H_0$, the data provide convincing evidence to suggest that the rate of sleep deprivation for New Yorkers is higher than the rate of sleep deprivation in the population at large.}

\eoceSol{(a) $H_0: p = 0.18, H_A: p \ne 0.18$. All assumptions/conditions satisfied. Z = 0.71, p-value = 0.4778. Fail to reject $H_0$, the data do not provide convincing evidence to suggest that the percentage of students at this university who smoke has changed over the last five years. (b) Type II.}

\eoceSol{If in fact the true market share of Dunder Mifflin was 45\%, the probability of getting a sample of 180 businesses where 36\% or more of them have Dunder Mifflin as their sole paper provider would be 0.9925.}

\eoceSol{If in fact 65\% Kansas residents went out of state for college, the probability of getting a sample of 1,500 Kansas residents where more than 1,005 go out of state for college would be 0.0526.}

\eoceSol{If in fact aerocyte is not not associated with a higher risk of stomach cancer, the probability of finding a random sample of 300 individuals where 9\% or more have stomach cancer is 0.0007.}

\eoceSol{Independence is satisfied since both groups are randomly samples and both samples are less than 10\% of the prospective populations. There are at least 10 successes and 10 failures in both samples and the samples are independent of each other. (0.16,0.22) (b) We are 90\% confident that the true proportion of Kansas residents that go out of state for college is 16\% to 22\% higher than the true proportion of California residents who go out of state for college. (c) Yes. (d) No.}

\eoceSol{(a) $H_0: p_D = p_I $ (Proportions of Democrats and independents who support the plan are equal.), $H_0: p_D > p_I $ (Proportion of Democrats who support the plan is higher than the proportion of independents who support the plan.) (b) Z = 11.32 (c) The probability of observing a $z$-statistic of 11.32 or higher given that the proportion of Democrats and independents who support the plan were equal is very low, almost 0. (d) Reject $H_0$, the data provide convincing evidence to suggest that the proportion of Democrats who support the plan is higher than the proportion of independents who support the plan. (e) Type I. (f) No. (g) (0.24, 0.32) (h) We are 90\% confident that the proportion of Democrats who support the plan is between 24\% and 32\% higher than the proportion of Independents who do. (i) 90\% of random samples of equivalent sizes will yield confidence interval that include the true difference between the population proportions of Democrats and Independents who support the plan. (f) No.}

\eoceSol{We are 95\% confident that the proportion of Californians who are sleep deprives is 1.7\% less to 0.1\% more than the proportion of Oregonians who are sleep deprived.}

\eoceSol{(a) False, proportion of males whose favorite color is black is higher than the proportion of females. (b) True. (c) True. (d) True. (e) False, (-0.06,-0.02).}

\eoceSol{(a) $P(Do~not~know | College~Grad) = 0.237, P(Do~not~know | Non~College~Grad) = 0.337$. (b) $H_0: p_{CG} = p_{NCG}, H_A: p_{CG} < p_{NCG}$. (c) Z = -3.18, p-value = 0.0007. (d) Reject $H_0$, the data provide convincing evidence to suggest that the proportion of college graduates who responded do not know is lower than the proportion of non-college graduates who responded do not know.}

\eoceSol{(a) $H_0: p_D = p_R$ (Proportions of Republicans and Democrats who support the use of full-body scans are equal.), $H_0: p_D \ne p_R $ (Proportions of Republicans and Democrats who support the use of full-body scans are different.) (b) Z = 0.68 (c) If in fact the proportions of Republicans and Democrats who support the use of full-body scans at airports were equal, the probability of getting a random sample of 318 Republicans and 318 Democrats where difference between the proportions in these samples is 2\% or higher would be 0.4966. (d) Fail to reject $H_0$, there isn't convincing evidence to suggest that the proportions of Republicans and Democrats who support the use of full-body scans are different. (e) Type II. (f) Yes.}

\eoceSol{(a) $H_0$: The distribution of the format of the book used by the students follows the professor's predictions. $H_A$: The distribution of the format of the book used by the students does not follow the professor's predictions. (b) $E_{hard~copy} = 75.6, E_{print} = 31.5, E_{online} = 18.9$. (c) Independence is satisfied assuming random sampling and since the sample size is less than 10\% of population. All expected counts are at least 10. Format of the book used is a categorical variable. (d) $\chi^2  = 2.32, df = 2, 0.01 < p-value < 0.02$. (e) Reject $H_0$, there is convincing evidence to suggest that the distribution of the format of the book used by the students does not follow the professor's predictions, i.e. the professor seems to be off in her predictions.}

\eoceSol{(a) 47.5 (b) 296.6 (c) 21}

\eoceSol{(a) Table below. (b) $E_K = 875.5, E_C = 729.5$
{\small
\begin{center}
\begin{tabular}{l l c c c}
								&			& \multicolumn{2}{c}{\textit{Went out of state}}	&		\\
\cline{3-4}
								&			& Yes		& No		& Total	\\
\cline{2-5}
\multirow{2}{*}{\textit{State	}}		& Kansas		& 1,005		& 495	& 1,500	\\
								& California	& 600		& 650	& 1,250	\\
\cline{2-5}
								& Total		& 1,605		& 1,145	& 2,750
\end{tabular}
\end{center} 
}
}

\eoceSol{(a) Table below. (b) $H_0$: There is no difference in virologic failure rates between the Nevaripine and Lopinavir groups, $H_A$: There is some difference in virologic failure rates between the Nevaripine and Lopinavir groups. (c) $E_{row~1, col~1} = 18, E_{row~1, col~2} = 102, E_{row~2, col~1} = 18, E_{row~2, col~2} = 102$. (d) Yes. (e) $\chi^2  = 8.37, df = 1, 0.001 < p-value < 0.005$. (f) Reject $H_0$, there is convincing evidence to suggest a difference in virologic failure rates between the Nevaripine and Lopinavir groups, i.e treatment and virologic failure do not appear to be independent.
{\small
\begin{center}
\begin{tabular}{l l c c c}
								&			& \multicolumn{2}{c}{\textit{Virol. failure}}	&		\\
\cline{3-4}
								&			& Yes		& No		& Total	\\
\cline{2-5}
\multirow{2}{*}{\textit{Treatment}}		& Nevaripine	& 26			& 94		& 120	\\
								& Lopinavir	& 10			& 110	& 120	\\
\cline{2-5}
								& Total		& 36			& 204	& 240
\end{tabular}
\end{center} 
}
}

\eoceSol{(a) Table below. (b) $H_0$: There is no difference in rate of quitting smoking between smokers who use a nicotine patch who have and have not been part of a support group., $H_A$: There is some difference in rate of quitting smoking between smokers who use a nicotine patch who have and have not been part of a support group. (c) i. 35 ii. 115 (d) Yes. (e) $\chi^2  = 1.86, df = 1, 0.1 < p-value < 0.2$. Fail to reject $H_0$, the study does not provide convincing evidence to support the claim that being in a support group changes the success rate for quitting smoking when someone is wearing a nicotine patch.
{\small
\begin{center}
\begin{tabular}{l l c c c}
								&			& \multicolumn{2}{c}{\textit{Quitting}}	&		\\
\cline{3-4}
								&			& Quit		& Not quit					& Total	\\
\cline{2-5}
\multirow{2}{*}{\textit{Support}}			& Support		& 40			& 110					& 150	\\
								& No support	& 30			& 120 					& 150	\\
\cline{2-5}
								& Total		& 70			& 230					& 300
\end{tabular}
\end{center}
}
}

\eoceSol{(a) $H_0$: There is no difference in rates of preferred shipping method and age among Los Angeles residents, $H_A$: There is some difference in rates of preferred shipping method and age among Los Angeles residents. (b) Not all expected counts are at least 10. (c) No, not all assumptions are met. See Chapter 6 for an alternative approach to test for independence using this data set.}

%\end{multicols}

\end{document}  